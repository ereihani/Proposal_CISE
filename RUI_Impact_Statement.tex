\documentclass[12pt]{article}
\usepackage[margin=1in]{geometry}
\usepackage{amsmath}
\usepackage{amsfonts}
\usepackage{amssymb}
\usepackage{graphicx}
\usepackage{cite}
\usepackage{url}
\usepackage{setspace}
\usepackage{fancyhdr}
\usepackage{titlesec}
\usepackage{enumitem}
\usepackage{float}
\usepackage[dvipsnames]{xcolor}
\usepackage[utf8]{inputenc}
\usepackage[T1]{fontenc}

\setstretch{1.0}

\begin{document}

\title{RUI Impact Statement}
\author{}
\date{}
\maketitle

This RUI project will establish a signature research thrust in resilient microgrid control at California State University, Bakersfield (CSUB), a predominantly undergraduate institution, by building an end-to-end, vendor-agnostic bump-in-the-wire (BITW) testbed that fuses physics-informed machine learning, distributed optimization, and control-barrier-function safety. The work positions CSUB's Electrical \& Computer Engineering program as a regional hub for clean-energy cyber-physical systems and strengthens the California Energy Research Center (CERC) as the locus for partnerships, demonstrations, and student training. By openly releasing scenario files, controller weights, signed logs, and reproducible plotting scripts, the project creates durable institutional assets that elevate CSUB's research profile and make replication straightforward across courses and capstone teams.

The award builds department capacity in three complementary ways. First, it delivers a delay-aware microgrid research testbed—spanning software and BITW edge hardware—that can impose realistic communication conditions such as variable delay and packet loss while tracking stability and safety metrics suitable for undergraduate research and instruction. Second, it generates reusable curricular modules that connect power systems, control, and AI/ML—covering physics-informed neural ODEs, multi-agent coordination, and optimization—that feed directly into power systems, embedded systems, and data-driven control courses. Third, it creates a sustained pipeline of capstones and independent studies aligned with regional utility and distributed-energy-resource integrator needs, coordinated through CERC for MOUs, pilots, and permitting-related collaborations. Together these elements form a durable ecosystem at a PUI: students contribute to publishable research, faculty gain modern infrastructure to sustain external funding, and the department differentiates its graduates for a high-growth clean-energy workforce.

For the PI, the project consolidates and accelerates a coherent research agenda in cyber-physical energy systems anchored in formal safety, distributed coordination, and physics-informed learning. The proposed architecture—cloud training flowing to edge inference with multi-agent coordination—and the validation plan—delay-robust stability, invariance guarantees, and cost impacts—create a stream of publishable theory-plus-systems results, expand collaborations with utilities, national labs, and industry, and establish a mentored undergraduate research team that can grow year over year. Clear success metrics, ablations, and benchmarking against accepted baselines strengthen CSUB's credibility in cyber-physical systems and energy research while positioning the PI for competitive follow-on NSF and DOE proposals.

Undergraduates are the core contributors and beneficiaries. The research is scaffolded so students progressively assume leadership across data pipelines and validation, embedded optimization at the edge, multi-agent consensus and federated learning, and safety verification. The project intentionally recruits women, first-generation students, and students from groups historically underrepresented in engineering, providing authentic research experiences that translate to internships, graduate admission, and industry roles in AI-enabled energy systems. Students gain hands-on experience with physics-informed neural ODEs and real-time embedded control on BITW hardware with tight inference budgets, and they are immersed in a verification-first culture that uses signed logs, fixed seeds, and documented scenarios to ensure reproducibility. Weekly one-on-one mentoring, professional-development workshops, and opportunities to co-author papers and present at conferences are built into the plan.

Research and education are tightly integrated. The BITW testbed and code artifacts become laboratory exercises in power systems, control, embedded systems, and AI/ML courses, where students explore delay-aware control, safety filtering via control barrier functions, and distributed optimization on real hardware. CERC will coordinate live demonstrations and short skills workshops open to students across majors, and the public release of controller weights, datasets, and scripts lowers the barrier for students to reproduce figures, extend experiments, and transition class projects into publishable results.

The project's location in California's Central Valley, serving a diverse student body, magnifies its broader impacts. By aligning research with local grid-resilience needs and creating visible, mentored roles for underrepresented students, the work forms a clean-energy workforce pipeline that benefits regional utilities and integrators. Through CERC, the team will engage partners for pilot demonstrations and capstone sponsorships, increasing student visibility and employability while informing the research with real operational constraints.

Dissemination and sustainability are addressed through permissively licensed releases of code, ONNX models, signed logs, and documented experiment scenarios, along with presentations at cyber-physical systems and energy venues and at undergraduate research symposia. Sustainability comes from folding the testbed into required labs and recurring capstones, leveraging early results to pursue follow-on NSF, DOE, and industry funding, and formalizing utility-sponsored student projects through CERC to create a repeating cycle of mentored undergraduate research.

Assessment is built around annual quantitative and qualitative metrics: the number and demographics of student researchers; the extent to which artifacts are reused in courses; capstones completed; internships and graduate placements; presentations and publications; and the initiation of external partnerships. Technical key performance indicators mirror the proposal's evidence map—stability and safety metrics, convergence targets, and total-cost-of-ownership impacts—reinforcing an outcomes-driven culture that advances institutional capacity, elevates faculty scholarship, and delivers high-value undergraduate research experiences.

\end{document}