\documentclass[11pt]{article}
\usepackage[margin=1in]{geometry}
\usepackage{amsmath}
\usepackage{setspace}
\usepackage[utf8]{inputenc}
\usepackage[T1]{fontenc}

% NSF-compliant line spacing (max 6 lines per inch)
\setstretch{1.15}
\setlength{\parindent}{0pt}
\setlength{\parskip}{3pt}

\begin{document}

\textbf{\large CPS-FR: RUI: Physics-Informed Machine Learning for Resilient Microgrid Control}

\textbf{Principal Investigator:} Ehsan Reihani

\section*{Overview}

Modern renewable-heavy microgrids face a critical challenge: conventional controllers become unreliable and expensive as network delays and packet loss cause instability that can cascade into blackouts. Our solution is a vendor-agnostic bump-in-the-wire controller that integrates with any existing equipment, eliminating vendor lock-in while dramatically improving performance. The system combines physics-informed machine learning, real-time edge computing under 10 ms, and intelligent multi-agent coordination. This achieves remarkable resilience---tolerating delays up to 150 ms with 20\% packet loss---while maintaining IEEE standards. Testing demonstrates frequency deviations under 0.30 Hz, 43-second recovery, zero violations, 30\% faster response, and 82\% cost savings.

\section*{Intellectual Merit}

This research makes four key contributions to resilient microgrid control. We develop Physics-Informed Neural Networks that learn power system behavior while respecting physical laws, providing stability guarantees under communication delays. Our multi-agent coordination algorithm enables intelligent grid component cooperation, achieving exponential convergence despite network disruptions. We create graph neural network optimization solving power flow problems in 17 iterations with under 10 ms runtime for real-time control. Finally, safety barrier functions automatically prevent dangerous conditions during storms, failures, or cyberattacks. The device installs between existing equipment without replacement, enabling incremental utility upgrades with reproducible benchmarks and open-source implementation.

\section*{Broader Impacts}

This project delivers a vendor-agnostic "bump-in-the-wire" controller that drops into existing equipment, speeding commissioning, eliminating vendor lock-in, and cutting total cost of ownership for hospitals, campuses, and emergency services. Economic analyses and Monte Carlo studies indicate ~82\% total cost of ownership savings with a $\leq$2-year payback. To move results into practice, CSUB's California Energy Research Center (CERC) will coordinate partners and pilots, run demonstrations, manage agreements and permitting, and support student capstones and short, skills-focused training.We will release open, versioned artifacts—code, datasets, benchmarks, ONNX models, reference hardware, and commissioning guides—under permissive licenses to speed adoption by utilities, integrators, and small businesses. These actions shorten adoption timelines, reduce downtime risk, and accelerate standards-aligned DER integration across critical infrastructure.

\end{document}