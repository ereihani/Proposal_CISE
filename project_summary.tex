\documentclass[10pt]{article}
\usepackage[margin=0.5in]{geometry}
\usepackage{amsmath}
\usepackage[utf8]{inputenc}
\usepackage[T1]{fontenc}

% Ultra-compact formatting
\setlength{\parindent}{0pt}
\setlength{\parskip}{4pt}

\begin{document}

\textbf{\large Project Summary: Physics-Informed Machine Learning for Resilient Microgrid Control}

\textbf{Principal Investigator:} Ehsan Reihani

\section*{Overview}

This CPS-FR project tackles a central barrier to reliable, renewable-heavy microgrids: conventional controllers are costly, vendor-locked, and prone to instability when real networks impose 50--100 ms delays and packet loss. We propose a \textbf{vendor-agnostic bump-in-the-wire (BITW) controller} that integrates physics-informed machine learning with multi-agent coordination to preserve stability and safety under adverse communications while cutting total cost of ownership. The three-layer architecture---cloud (federated training), edge (real-time inference $<10$ ms), and multi-agent coordination (consensus/optimization)---targets \textbf{delay tolerance to 150 ms with up to 20\% packet loss}, compliance with IEEE 1547/2030.7/2030.8, and elimination of proprietary lock-in. Preliminary results on a 16-agent setup show \textbf{max $|\Delta f| \leq 0.30$ Hz}, \textbf{$\approx$43 s settling}, and \textbf{0 violations in a 30-minute run}; overall, we observe $\sim$30\% faster convergence and an \textbf{$\approx$82\% cost reduction} relative to conventional approaches. Validation scales scenarios and benchmarks and ties claims to fixed pass/fail gates and artifacts.

\section*{Intellectual Merit}

The BITW controller offers a unified, formally grounded framework that advances the state of the art in cyber-physical microgrid control. (1) \textbf{Physics-Informed Neural ODEs} embed power-system dynamics in the learning objective, delivering delay-dependent ISS guarantees; (2) \textbf{Multi-Agent Reinforcement Learning with consensus dynamics} provides exponential convergence to coordinated setpoints under realistic delays; (3) \textbf{GNN-accelerated distributed optimization (ADMM)} achieves $\sim$17 iterations to 1\% optimality with edge-runtime feasibility ($<10$ ms p95/iteration); and (4) \textbf{Control Barrier Functions} enforce forward-invariant safety for frequency/voltage during disturbances and N-1/N-2 events. These elements are integrated into a vendor-agnostic BITW device that sits between existing infrastructure and inverters, enabling standardized protocols and safety enforcement without replacing incumbent equipment. The evaluation plan maps each claim to specific metrics and evidence locations (e.g., $|\Delta f|$, settling time, violations/hour; iterations and p95 latency; TCO/payback), with binary gates for go/no-go decisions. A first-class reproducibility package (code, seeds, signed logs, ONNX weights, provenance tags, recurring CI rebuilds) ensures external groups can regenerate figures and tables from raw traces.

\section*{Broader Impacts}

Reliable, affordable microgrids are essential to hospitals, campuses, emergency services, and other critical infrastructure. By \textbf{reducing controller cost and eliminating vendor lock-in}, the proposed BITW approach lowers adoption barriers and directly supports resilient clean-energy deployment. Economic analyses and Monte Carlo studies indicate \textbf{$\sim$82\% TCO savings} with \textbf{$\leq$2-year payback}, aligning with national goals to reduce the heavy economic burden of sustained outages and to scale standardized, interoperable microgrid solutions. The project embeds \textbf{workforce development and broadening participation} through structured, mentored roles for undergraduate researchers from underrepresented groups, spanning data pipelines, embedded systems, distributed optimization, and security---skills transferable to high-growth clean-energy and AI sectors. All artifacts will be openly released with durable identifiers to accelerate technology transfer to utilities, communities, and industry partners.

\end{document}