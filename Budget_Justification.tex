\documentclass[12pt]{article}
\usepackage[margin=1in]{geometry}
\usepackage{amsmath}
\usepackage{amsfonts}
\usepackage{amssymb}
\usepackage{graphicx}
\usepackage{cite}
\usepackage{url}
\usepackage{setspace}
\usepackage{fancyhdr}
\usepackage{titlesec}
\usepackage{enumitem}
\usepackage{float}
\usepackage{xcolor}
\usepackage[utf8]{inputenc}
\usepackage[T1]{fontenc}
\usepackage{tabularx}

% Set line spacing
\onehalfspacing

% Section formatting
\titleformat{\section}{\large\bfseries}{\thesection}{1em}{}
\titleformat{\subsection}{\normalsize\bfseries}{\thesubsection}{1em}{}
\titleformat{\subsubsection}{\normalsize\bfseries}{\thesubsubsection}{1em}{}

\begin{document}

\title{\Large\textbf{Budget Justification\\CPS-FR: RUI: Physics-Informed Machine Learning for Resilient Microgrid Control}}
\author{Principal Investigator: Ehsan Reihani\\California State University, Bakersfield}
\date{}

\maketitle

\section{A. SENIOR PERSONNEL}

\textbf{Principal Investigator: Professor Ehsan Reihani}

Professor Reihani will serve as the Principal Investigator for this four-year project. His annual salary is \$116,100 based on an 8-month academic calendar. The PI will dedicate 1.6 academic months per year (equivalent to 20\% effort annually) to this research project.

During the project period, Professor Reihani will:
\begin{itemize}
    \item Lead the development of physics-informed machine learning algorithms for microgrid control
    \item Supervise and mentor 3 undergraduate student researchers annually
    \item Coordinate with industry partners and the California Energy Research Center (CERC)
    \item Oversee experimental validation and safety verification procedures
    \item Disseminate research findings through peer-reviewed publications and conference presentations
\end{itemize}

The PI's compensation includes fringe benefits calculated at the university's composite rate of 48\% for academic year appointments.

\section{B. OTHER PERSONNEL}

\textbf{Undergraduate Student Researchers}

This project will support 3 undergraduate student researchers annually, providing critical hands-on experience in emerging technologies at the intersection of artificial intelligence, control systems, and clean energy. Our Kern County location provides unique opportunities to engage women and individuals from underrepresented groups in cutting-edge research.

\textbf{Academic Year Support:}
\begin{itemize}
    \item 3 students × 20 hours/week × \$20/hour × 30 weeks = \$36,000 annually
    \item Students will work on data pipeline development, embedded systems programming, distributed optimization algorithms, and real-time control implementation
\end{itemize}

\textbf{Summer Support:}
\begin{itemize}
    \item 3 students × 20 hours/week × \$20/hour × 15 weeks = \$18,000 annually
    \item Summer fringe benefits at 16\% composite rate = \$2,880 annually
    \item Total summer support: \$20,880 annually
\end{itemize}

The comprehensive research experience spans multiple STEM disciplines---power systems engineering, machine learning, optimization theory, and cyber-physical systems---creating educational pathways that bridge traditional engineering with cutting-edge computational sciences.

\section{C. FRINGE BENEFITS}

Fringe benefits are calculated using California State University, Bakersfield's established composite rates:

\begin{itemize}
    \item \textbf{Faculty Academic Year:} 48\% composite rate applied to PI salary
    \item \textbf{Summer Personnel (Faculty/Staff/Students):} 16\% composite rate applied to summer wages
\end{itemize}

These rates include health insurance, retirement contributions, social security, Medicare, unemployment insurance, workers' compensation, and other mandatory benefits as established by the university's financial policies.

\section{D. EQUIPMENT}

Not applicable. This project will utilize existing laboratory facilities and leverage industrial-grade NVIDIA Jetson AGX Orin hardware at \$1,999 per 64GB production module, which is categorized under materials and supplies rather than permanent equipment.

\section{E. TRAVEL}

\textbf{Domestic Conference Travel}

Research dissemination and professional development are critical components of this project. The travel budget supports:

\textbf{Principal Investigator:}
\begin{itemize}
    \item 3 conferences per year × 4 years = 12 trips
    \item Cost per trip: \$2,700 (includes airfare, lodging, conference registration, per diem, and ground transportation)
    \item PI travel total: \$32,400 over 4 years
\end{itemize}

\textbf{Undergraduate Student Researchers:}
\begin{itemize}
    \item 3 students × 1 conference each per year × 4 years = 12 student trips
    \item Cost per student trip: \$2,700
    \item Student travel total: \$32,400 over 4 years
\end{itemize}

\textbf{Total Domestic Travel: \$64,800 over 4 years}

Target conference venues include IEEE Power \& Energy Society General Meeting, American Control Conference, IEEE Conference on Decision and Control, ACM/IEEE International Conference on Cyber-Physical Systems, and machine learning conferences relevant to physics-informed neural networks. These venues provide optimal opportunities to present research findings, collaborate with leading experts, and expose undergraduate researchers to cutting-edge developments in the field.

\section{F. PARTICIPANT SUPPORT COSTS}

Not applicable to this research project.

\section{G. OTHER DIRECT COSTS}

\subsection{1. Materials and Supplies}

\textbf{\$10,000 per year (\$40,000 total over 4 years)}

This budget category covers essential research materials including:
\begin{itemize}
    \item Computing peripherals and hardware adapters for embedded systems integration
    \item Electronic test equipment and measurement devices for microgrid validation
    \item Hardware components for bump-in-the-wire controller prototypes
    \item Cloud computing credits for large-scale distributed training and simulation
    \item Laboratory consumables and research supplies
\end{itemize}

The vendor-agnostic approach requires diverse hardware integration capabilities, necessitating specialized adapters and testing equipment to ensure compatibility across different microgrid configurations.

\subsection{2. Publication Costs}

\textbf{\$5,000 in Year 3}

Open access publication fees for high-impact peer-reviewed journals in the fields of:
\begin{itemize}
    \item IEEE Transactions on Smart Grid
    \item IEEE Transactions on Power Systems
    \item IEEE Transactions on Control Systems Technology
    \item Applied Energy
    \item Journal of Machine Learning Research
\end{itemize}

This investment ensures maximum dissemination of research findings and contributes to the broader scientific community's understanding of physics-informed machine learning for power system applications.

\subsection{3. Computer Services}

\textbf{\$10,000 per year (\$40,000 total over 4 years)}

Advanced computational requirements for this project include:
\begin{itemize}
    \item Cloud computing resources for federated learning and distributed optimization algorithms
    \item AI/ML services for physics-informed neural network training and validation
    \item Software licenses for specialized power system simulation tools and development environments
    \item High-performance computing resources for large-scale Monte Carlo analysis and safety verification
    \item Data storage and management services for experimental datasets and reproducibility artifacts
\end{itemize}

The three-layer architecture (cloud, edge, multi-agent coordination) requires substantial computational resources to achieve the target performance metrics of 150 ms delay tolerance with 20\% packet loss while maintaining mathematical safety guarantees.

\section{H. INDIRECT COSTS}

Indirect costs are calculated at California State University, Bakersfield's federally negotiated rate of 48\% applied to the Modified Total Direct Cost (MTDC) base. This rate covers institutional overhead expenses including:

\begin{itemize}
    \item Administrative support and sponsored programs management
    \item Facilities operation and maintenance
    \item Library resources and information technology support
    \item General institutional expenses related to research administration
\end{itemize}

The MTDC base excludes equipment costs over \$5,000, participant support costs, and the portion of each subaward exceeding \$25,000, in accordance with federal guidelines.

\vspace{0.5cm}

\textbf{Budget Summary}

This budget justification demonstrates efficient resource allocation to achieve the project's ambitious goals of developing vendor-agnostic microgrid control technology while providing transformational educational experiences for undergraduate researchers. The 82\% cost reduction and 2-year payback period projected for the resulting technology will deliver significant return on this research investment, supporting national energy resilience and clean technology deployment priorities.

\end{document}